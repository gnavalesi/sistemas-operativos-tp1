\section{Ejercicio 1}

Definimos como $item$ a un par $(string, entero)$ y como $bucket$ a una lista enlazada de $item$s.


\subsection{Estructura de ConcurrentHashMap}

La clase cuenta con un miembro público \mintinline{c++}{bucket* tabla[26]} que guardará los buckets correspondientes a cada clave. En un miembro privado \mintinline{c++}{mutex *bucket_mutex[26]} se almacenará por cada bucket un mutex que permitirá aplicar exclusión mutua sobre ellos (ver \cite{cppref-mutex}).

Otros miembros privados son \mintinline{c++}{unsigned int hash_key(string *key)}, que mapea la primera letra de \mintinline{c++}{key} a $\{ 0, 1, \cdots, 25 \}$ de forma biyectiva, \mintinline{c++}{struct}s de parámetros y funciones de threads, y otras funciones auxiliares.


\subsection{Constructor y destructor}

El constructor simplemente crea los 26 buckets y los correspondientes mutexs y el destructor los elimina liberando su memoria.


\subsection{\mintinline{c++}{void addAndInc(string key)}}

\begin{minted}{c++}
void addAndInc(string key) {
	Obtener el bucket correspondiente a la clave
	Activar el lock del bucket
	Buscar la clave en el bucket
	Si se encuentra la clave, incrementar la cuenta
	Si no se encuentra la clave, agregarlo al bucket con la cuenta en 1
	Desactivar el lock del bucket
}
\end{minted}

El locking del bucket antes de realizar la búsqueda e incrementar/iniciar el contador garantiza que sólo exista contención en caso de colisión de hash.


\subsection{\mintinline{c++}{bool member(string key)}}

\begin{minted}{c++}
bool member(string key) {
	Obtener el bucket correspondiente a la clave
	Buscar la clave en el bucket
	Devolver si se encuentra la clave
}
\end{minted}

Como no hay locking en la función, esta es wait-free.


\subsection{\mintinline{c++}{item maximum(unsigned int nt)}}

La función crea threads que ejecutan la función \mintinline{c++}{maximum_thread_function} y reciben sus argumentos a través de la estructura \mintinline{c++}{maximum_thread_args}.

\begin{minted}{c++}
item maximum(unsigned int nt) {
	atomic<int> index = 0;
	item maximo_global = item("", 0);
	mutex maximo_global_mutex;
	Activar el lock de todos los buckets
	Crear nt threads {
		int i = index.getAndInc();
		Mientras i < 26 {
			Guardar en maximo_local el item con la máxima cuenta del bucket i
			i = index.getAndInc();
		}
		Si se encontró un maximo_local {
			Activar el lock de maximo_global
			Actualizar maximo_global con el máximo entre maximo_global y maximo_local
			Desactivar el lock de maximo_global
		}
	}
	Esperar a que terminen los threads
	Desactivar el lock de todos los buckets
	Devolver general_maximum
}
\end{minted}

El uso de \mintinline{c++}{atomic<int> index} le permite a cada thread obtener un único índice correspondiente a un bucket sin procesar en cada iteración del ciclo.

El locking de todos los buckets antes de buscar el máximo garantiza que la búsqueda no sea concurrente con la función \mintinline{c++}{void addAndInc(string key)}.

Los threads procesan buckets únicos hasta que no quede ninguno. Cuando a un thread no le quedan buckets, si encontró un máximo local, se activa el locking de la variable compartida que guarda el valor máximo global y se actualiza su valor. Esto evita las condiciones de carrera sobre esta variable.