\section{Ejercicio 1}

Definimos como $item$ a un par $(string, entero)$ y como $bucket$ a una lista enlazada de $item$s.


\subsection{Estructura de ConcurrentHashMap}

La clase cuenta con un miembro público \mintinline{c++}{bucket* tabla[26]} que guardará los buckets correspondientes a cada clave. En un miembro privado \mintinline{c++}{mutex *tabla_mutex[26]} se almacenará por cada bucket un mutex que permitirá aplicar exclusión mutua sobre ellos (ver \cite{cppref-mutex}).

Otros miembros privados son \mintinline{c++}{unsigned int hash_key(string *key)}, que mapea la primera letra de \mintinline{c++}{key} a $\{ 0, 1, \cdots, 25 \}$ de forma biyectiva, \mintinline{c++}{struct}s de parámetros y funciones de threads, y otras funciones auxiliares.


\subsection{Constructor y destructor}

El constructor simplementa crea los 26 buckets y los correspondientes mutexs y el destructor los elimina liberando su memoria.


\subsection{\mintinline{c++}{void addAndInc(string key)}}

\begin{minted}[tabsize=4]{c++}
void addAndInc(string key) {
	unsigned int index = hash_key(&key);
	// Activo locking el bucket
	tabla_mutex[index]->lock();
	// Busco la clave
	bucket::Iterador it = tabla[index]->CrearIt();
	while (it.HaySiguiente() && it.Siguiente().first != key) it.Avanzar();
	// Si existe, incremento la cuenta. Si no, la inicio en 1.
	if (it.HaySiguiente()) it.Siguiente().second++;
	else tabla[index]->push_front(item(key.data(), 1));
	// Desactivo locking el bucket
	tabla_mutex[index]->unlock();
}
\end{minted}

El locking del bucket antes de realizar la búsqueda e incrementar/iniciar el contador garantiza que sólo exista contención en caso de colisión de hash.


\subsection{\mintinline{c++}{bool member(string key)}}

\begin{minted}[tabsize=4]{c++}
bool member(string key) {
	unsigned int index = hash_key(&key);
	// Busco la clave
	bucket::Iterador it = tabla[index]->CrearIt();
	while (it.HaySiguiente() && it.Siguiente().first != key) it.Avanzar();
	// Devuelvo si la encontré o no
	return it.HaySiguiente();
}
\end{minted}

Como no hay locking en la función, esta es wait-free.


\subsection{\mintinline{c++}{item maximum(unsigned int nt)}}

La función crea threads que ejecutan la función \mintinline{c++}{maximum_thread_function} y reciben sus argumentos a través de la estructura \mintinline{c++}{maximum_thread_args}.

\begin{minted}[tabsize=4]{c++}
struct maximum_thread_args {
	ConcurrentHashMap *map = nullptr; // Puntero al ConcurrentHashMap
	atomic<int> index; // Índice del siguiente bucket a procesar
	item general_maximum; // Item con máxima cuenta encontrado
	mutex general_maximum_mutex; // Mutex para general_maximum
};
\end{minted}

\begin{minted}[breaklines, tabsize=4]{c++}
item maximum(unsigned int nt) {
	pthread_t threads[nt];
	// Preparo los argumentos de maximum_thread_function
	maximum_thread_args args;
	args.map = this;
	args.index = 0;
	args.general_maximum = item("", 0);
	// Creo los threads
	for (unsigned int i = 0; i < nt; i++)
		pthread_create(&threads[i], nullptr, maximum_thread_function, &args);

	// Espero por cada uno de los threads
	for (unsigned int i = 0; i < nt; i++)
		pthread_join(threads[i], nullptr);
	// Devuelvo el máximo
	return args.general_maximum;
}
\end{minted}

\begin{minted}[breaklines, tabsize=4]{c++}
void *maximum_thread_function(void *thread_args) {
	struct maximum_thread_args *args = thread_args;
	item maximum = item("", 0);
	int index;
	// Obtengo e incremento el índice del siguiente bucket atómicamente
	for (index = args->index++; index < 26; index = args->index++) {
		// Busco el máximo haciendo locking del bucket
		args->map->tabla_mutex[index]->lock();
		bucket::Iterador it = args->map->tabla[index]->CrearIt();
		for (it; it.HaySiguiente(); it.Avanzar()) {
			if (it.Siguiente().second > maximum.second)
				maximum = it.Siguiente();
		}
		args->map->tabla_mutex[index]->unlock();
	}
	// Actualizo el máximo si encontré uno
	if (maximum != item("", 0)) {
		args->general_maximum_mutex.lock();
		if (maximum.second > args->general_maximum.second)
			args->general_maximum = maximum;
		args->general_maximum_mutex.unlock();
	}
	return nullptr;
}
\end{minted}

El uso de \mintinline{c++}{atomic<int> index} en el ciclo permite obtener el valor de \mintinline{c++}{args->index} e incrementarlo de forma atómica (ver \cite{cppref-atomic}), lo que garantiza que cada thread obtenga un índice único.

El locking del bucket antes de buscar el máximo hace que la búsqueda no sea concurrente con la función \mintinline{c++}{void addAndInc(string key)}.

El ciclo hace que los threads sigan procesando buckets si queda alguno sin procesar. Cuando a un thread no le quedan buckets, si encontró un máximo local, se activa el locking de la variable compartida que guarda el valor máximo global y se actualiza su valor. Esto evita las condiciones de carrera sobre esta variable.