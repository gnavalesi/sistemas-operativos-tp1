\section{Ejercicio 2}


\subsection{\mintinline{c++}{void push_front(const T& val)}}

\begin{minted}[tabsize=4]{c++}
void push_front(const T& val) {
	Nodo *newHead = new Nodo(val);
	newHead->_next = _head.exchange(newHead);
}
\end{minted}

Como \mintinline{c++}{_head.exchange(newHead)} cambia el valor de \mintinline{c++}{_head} por el de \mintinline{c++}{newHead} y devuelve su valor anterior de forma atómica, no hay condiciones de carrera sobre esta variable y \mintinline{c++}{newHead} pasa a ser el primer elemento de la lista. Dado que la asignación de \mintinline{c++}{newHead->_next} no es atómica, no se puede garantizar que inmediatamente después de la operación anterior \mintinline{c++}{newHead} esté enlazada al valor que tenía \mintinline{c++}{_head} previo al \mintinline{c++}{exchange}. Es posible que en este período se agregue otro nodo delante de \mintinline{c++}{newHead}. Pero dado que \mintinline{c++}{_head.exchange(newHead)} devuelve el valor anterior de \mintinline{c++}{_head} atómicamente, la asignación lo guardará en \mintinline{c++}{newHead->_next} eventualmente y los nodos quedarán correctamente enlazados.


\subsection{\mintinline{c++}{ConcurrentHashMap count_words(string arch)}}

\begin{minted}{c++}
ConcurrentHashMap count_words(string arch) {
	ConcurrentHashMap map;
	count_words(arch, &map);
	return map;
}
\end{minted}

\begin{minted}{c++}
void count_words(string arch, ConcurrentHashMap *map) {
	list<string> palabras = words(arch);
	list<string>::iterator it;
	for (it = palabras.begin(); it != palabras.end(); it++) {
		map->addAndInc(*it);
	}
}
\end{minted}

\begin{minted}{c++}
list<string> words(string arch) {
	list<string> palabras;
	archivo = abrir_archivo(arch);
    while (archivo.quedan_palabras()) {
        palabras.push_back(archivo.leer_palabra());
        archivo.avanzar_palabra();
    }
    return palabras;
}
\end{minted}